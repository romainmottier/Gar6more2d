\documentclass[11pt,a4]{article}
\usepackage{a4wide}
%\usepackage{pstcol}
\usepackage{pstricks,pst-node,pst-text,pst-3d} 
\usepackage{comm}
\usepackage[latin1]{inputenc}
\title{Documentation for Gar6more2D}
\author{Julien Diaz and Abdelaaziz Ezziani}
\begin{document}
\maketitle
This code compute the quasi-analytical solution of
several wave equation in two layered media, using
the Cagniard de Hoop
method~\cite{Cag,DH,VDH,PG,QG,JD,RAP_DE6509,Rap3}. It produces
seismograms at given points.

The equations can be written in the general form.  
\begin{eqnarray}
 A(y)\derp {^2 U}{t^2}-B(y) U=\delta({\bf x}-{\bf x_s})\,f(t), 
\quad x\in\setR,\,y\in\setR
\end{eqnarray}
where $A$ and $B$ are operators satisfying\\[10pt]
\begin{tabular}{l}
$A(y)=A^+, \, B(y)=B^+, \quad y>0$, \\
$A(y)=A^-, \, B(y)=B^-, \quad y<0.$
\end{tabular} 
\\[10pt]
The code analytically compute the Green  function $u$ of the problem
\begin{eqnarray}
 A(y)\derp {^2 u}{t^2}-B(y) u=\delta({\mathbf
   x}-\mathbf{x_s})\,\delta(t), \quad x\in\setR, y\in\setR
\end{eqnarray}
and convolves it with the source function $f$. You
can modify this function in the subroutine {\it
  lib/libgeneral/source.F90}. The convolution is
done by a numerical integration, that is why the
solution is only ``quasi-analytical''. You can
improve the accuracy of the solution by increasing the
number of intervals used for the integration (the
variable {\it Nint} in the data file {\it cdh2d.dat})
\section{Acoustic/acoustic}
The code computes a seismogram at points
$(x_i,y)_{i=1,Nx}$ of the pressure solution of the equations
\begin{eqnarray*}
 \derp {^2 P^+}{t^2}-{c^+}^2 \Delta P^+=\delta({\mathbf x}-\mathbf{x_s})f(t),\quad x\in\setR, y>0\\[10pt]
 \derp {^2 P^-}{t^2}-{c^-}^2 \Delta P^-=\delta({\mathbf x}-\mathbf{x_s})f(t),\quad x\in\setR, y<0
\end{eqnarray*}
We consider $\mathbf{x_s}=(0,h)$ and we use the transmission conditions :\\[10pt]
\begin{tabular}{|l} 
$P^+=P^-$, \\[10pt]
$\dsp \rho^+ \frac{\partial P^-}{\partial y}=\rho^- \frac{\partial P^+}{\partial y} .$
\end{tabular}\\
on the interface $y=0$.
The code also computes the velocity given by the relation~:
$$\derp {\mathbf V^\pm} t =-\frac{1}{\rho^\pm}{\mathbf \nabla} P^\pm.$$ 
If you  want to compute  the displacement $U$, it can be easily computed by replacing $f(t)$ by the primitive of the source function you are using. For instance, if you are using a Rickert, you'll have to consider a first derivative of a Gaussian for $f$.  
\section{Acoustic/elastodynamic(isotropic)}
The code computes a seismogram at point
$(x_i,y)_{i=1,Nx}$ of the pressure (in the fluid) and the
velocity (in the solid) solution of the equations
\begin{eqnarray}
 \derp {^2 P^+}{t^2}-{c^+}^2 \Delta P^+=\delta({\mathbf x}-\mathbf{x_s})f(t),\quad x\in\setR, y>0,\\[10pt]
 \derp {^2 \mathbf{V}^-}{t^2}-(\lambda^-+2\mu^-)
 {\mathbf \nabla} (\nabla \cdot {\mathbf{V}})+\mu^-\nabla\times(\nabla \times {\mathbf{V}^-})=0,\quad x\in\setR, y<0,
\end{eqnarray}
with  $\mathbf{x_s}=(0,h)$ and the transmission conditions\\[10pt]
\begin{tabular}{|l} 
$\dsp \derp {V^-_y}t=-\frac{1}{\rho^+}\derp {P^+}{y},\quad y=0$, \\[10pt]
$\dsp (\lambda^-+2\mu^-) \derp {V_y^-}{y}+\lambda^- \derp{V_x^-}{x}=\derp{P^+}{t}$\\[10pt]
$\dsp \derp{V_x^-}y+\derp {V_y^-} x=0.$
\end{tabular}\\[10pt]
on the interface $y=0$. 
The code  also computes the velocity in the fluid by using the relation
$$\derp {V^+} t =-\frac{1}{\rho^+}\nabla P^+.$$ 
Once again, if you want to compute the displacement $U$, it can be easily computed by replacing $f(t)$ by the primitive of the source function you are using.
\section{Acoustic/poroelastic(see~\cite{RAP_DE6509})}
The code computes  a seismogram at point
$(x_i,y)_{i=1,Nx}$ of the potential of velocity
$\chi$ and the displacement $U^+$ (in the fluid) and the
solid displacement $U_s^-$ (in the poroelastic medium) solution of the equations
$$ \begin{array}{lr}
\left|
 \begin{array}{l}
 \dsp \derp {^2 \chi^+}{t^2}-{c^+}^2 \Delta
 \chi^+=\delta({\mathbf
   x}-\mathbf{x_s})f(t),\\[10pt]
\dsp \derp {U^+} t =-\nabla \chi^+.
\end{array}
\right.
\end{array}$$
for $x\in\setR, y>0$ and 
$$\begin{array}{lr}
\left|
  \begin{array}{l}
\dsp  (1-\phi^-)\rho_s^- \derp {^2
  \mathbf U_s^-}{t^2}+\phi\rho_f^-\derp {^2
  \mathbf U_f^-
}{t^2}-(\lambda^-+2\mu^-) \nabla (\nabla \cdot
\mathbf U_s^-)+\mu^-\nabla\times(\nabla \times \mathbf U_s^-)+\beta\,\nabla P^-=0
,\\[10pt]
    \dsp  (1-a^-)\rho_f^- \derp {^2
  \mathbf U_s^-}{t^2}+a^-\rho_f^-\derp {^2
  \mathbf U_f^-}{t^2}+\nabla P^-=0\\[10pt]
\dsp \frac 1 {m^-} P^- +(\beta^--\phi^-) \nabla \cdot \mathbf U_s^-+ \phi^- \nabla \cdot \mathbf U_f^-=0
  \end{array}
\right.
 \end{array}$$
for $x\in\setR, y<0$, either with the open pore transmission
conditions (if parameter open is set to 1)\\[10pt] 
\begin{tabular}{|l} 
$\dsp \phi^-(U_{fy}^--U_{sy}^-)=U^+_y-U_{sy}^-,$\\[12pt]
$\dsp P^-=\rho^+\derp{\chi^+}{t},$\\[12pt]
$\dsp \big(\lambda^-+m^-\beta^-(\beta^--\phi^-)\big)\,\nabla\cdot \mathbf U_s^-+2\mu^-\derp{U_{sy}^-}{y}+m^-\beta^-\phi^-\nabla\cdot \mathbf U_f^-
=-\rho^+\derp{\chi^+}{t},$\\[12pt]
$\dsp\derp{U_{sx}^-}{y}+\derp{U_{sy}^-}{x}=0$,
\end{tabular}
\\[10pt]
or with the sealed pore transmission
conditions (if parameter open is set to 0)\\[10pt] 
\begin{tabular}{|l} 
$\dsp \phi^-(U_{fy}^--U_{sy}^-)=U^+_y-U_{sy}^-,$\\[12pt]
$U_{fy}^-=U_{sy}^-,$\\[12pt]
$\dsp \big(\lambda^-+m^-\beta^-(\beta^--\phi^-)\big)\,\nabla\cdot \mathbf U_s^-+2\mu^-\derp{U_{sy}^-}{y}+m^-\beta^-\phi^-\nabla\cdot \mathbf U_f^-
=-\rho^+\derp{\chi^+}{t},$\\[12pt]
$\dsp\derp{U_{sx}^-}{y}+\derp{U_{sy}^-}{x}=0$,
\end{tabular}
on the interface $y=0$.
The code
does not compute the fluid displacement and the
pressure in the poroelastic medium, but there is
no particular difficulty to do that. 
\begin{Remark}
  The pressure in the fluid satisfies the relation
  $$P^+=\rho^+\frac{\partial \chi}{\partial t}.$$
  Therefore, if you want to compute a seismogram
  of the pressure  and the velocities in the fluid
  and of the velocities in the solid, you have to
  replace $f(t)$ by the derivative of the source
  function you are using. If you want to consider
  a source of pressure, you'll have to
  replace $f(t)$ by the integral of the source
  function you are using.
\end{Remark}
\section{Poroelastic/poroelastic (see~\cite{Rap3})}
The code computes a seismogram at point
$(x_i,y)_{i=1,Nx}$ of the solid displacement $U$
solution of the equations
$$ \begin{array}{lr}
\!\!\!\!\left|
  \begin{array}{l}
\dsp  (1-\phi^+)\rho_s^+ \derp {^2
  \mathbf U_s^+}{t^2}+\phi^+\rho_f^+\derp {^2
  \mathbf U_f^+}{t^2}-\!(\lambda^++2\mu^+) \nabla (\nabla \cdot
\mathbf U_s^+)\!+\!\mu^+\nabla\times(\nabla \times \mathbf U_s^+)+\beta^+\nabla P^+\!=\!\nabla\delta({\mathbf x}-\mathbf{x_s}) F_s(t)
,\\[10pt]
    \dsp  (1-a^+)\rho_f^+ \derp {^2
  \mathbf U_s^+}{t^2}+a^+\rho_f^+\derp {^2
  \mathbf U_f^+}{t^2}+\nabla P^+=\nabla\delta({\mathbf x}-\mathbf{x_s}) F_s(t)\\[10pt]
\dsp \frac 1 {m^+} P^+ +(\beta^+-\phi^+) \nabla
\cdot \mathbf U_s^++ \phi^+ \nabla \cdot \mathbf U_f^+=\delta({\mathbf x}-\mathbf{x_s}) F_p(t)
  \end{array}
\right. 
\end{array}$$
for $x\in\setR, y>0$ and
$$ \begin{array}{lr}
\!\!\!\!\left|
  \begin{array}{l}
\dsp  (1-\phi^-)\rho_s^- \derp {^2
  \mathbf U_s^-}{t^2}+\phi^-\rho_f^-\derp {^2
  \mathbf U_f^-}{t^2}-(\lambda^-+2\mu^-) \nabla (\nabla \cdot
\mathbf U_s^-)+\mu^-\nabla\times(\nabla \times \mathbf U_s^-)+\beta^-\nabla P^-=0
,\\[10pt]
    \dsp  (1-a^-)\rho_f^- \derp {^2
  \mathbf U_s^-}{t^2}+a^-\rho_f^-\derp {^2
  \mathbf U_f^-}{t^2}+\nabla P^-=0\\[10pt]
\dsp \frac 1 {m^-} P^- +(\beta^--\phi^-) \nabla
\cdot \mathbf U_s^-+ \phi^- \nabla \cdot \mathbf U_f^-=0
  \end{array}
\right. 
\end{array}$$
for $x\in\setR, y<0$, with the transmission conditions on the interface $y=0$ \\[10pt]
\begin{tabular}{|l}
$\phi^-\,(U_{fy}^--U_{sy}^-)=\phi^+\,(U_{fy}^+-U_{sy}^+),$\\[12pt]
$\dsp \alpha^-\nabla\cdot \mathbf U_s^-
+2\mu^-\derp{ U_{sy}^-}{y}+m^-\beta^-\phi^-\,\nabla\cdot \mathbf U_f^-=
\alpha^+\nabla\cdot \mathbf U_s^+
+2\mu^+\derp{U_{sy}^+}{y} + m^+\beta^+\phi^+\,\nabla\cdot \mathbf U_f^+,$\\[12pt]
$\dsp \mu^-(\derp {U_{sx}^-}{y}+\derp{U_{sy}}{x}^-)=
\mu^+(\derp{ U_{sy}^+}{y} + \derp {U_{sy}^+}{x}),$\\[12pt]
$U_{sx}^-=U_{sx}^+,\quad  U_{sy}^-=U_{sy}^+,\quad P^-=P^+,$
\end{tabular} 
\\[10pt]
with 
$$
\alpha^\pm=\lambda^\pm+m^\pm\beta^\pm(\beta^\pm-\phi^\pm).
$$
Actually the code computes the solution for each
source $F_s$ and $F_p$ separately. If you want a
bulk source ($F_s$), set the parameter
$\it type\_source$ to 1, if you want a pressure source
($F_p$), set the parameter
$\it type\_source$ to 2. 
\begin{Remark}
  The code does not really compute the
  displacement, but its derivative (for some reasons
  related to the Cagniard-de Hoop method,
  see~\cite{ezziani_th,Rap3}). Therefore, you have to
  replace $f(t)$ by the primitive of the source
  function you are using to compute the displacement. 
\end{Remark}
\bibliographystyle{plain}
\bibliography{biblio} 
\end{document}
